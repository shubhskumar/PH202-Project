\documentclass[14pt]{extarticle}
\usepackage[margin=0.7in]{geometry}
\usepackage[utf8]{inputenc}
\usepackage{amsmath,amssymb}
\usepackage{physics}
\usepackage{textcomp}
\usepackage{natbib}
\usepackage{graphicx}

\title{PH202 Project}
\author{Shubhankar Kumar }
\date{March 2021}

\begin{document}

\maketitle

\section{Introduction to Mueller Matrices and Stokes vectors}
\subsection{Mueller Matrices}
Mueller matrices are 4x4 matrices that describe the linear relationship between polarization states of an incident light beam and the emerging light beam after passing through some polarizing element(s).
The light beam is characterized by its Stokes Parameters. If S = $\big(\begin{smallmatrix}
  S_0 & S_1 & S_2 & S_3\\
\end{smallmatrix}\big)^T$ are the stokes parameters of the incident light beam, and S' = $\big(\begin{smallmatrix}
  S_0' & S_1' & S_2' & S_3'\\
\end{smallmatrix}\big)^T$ are the stokes parameters of the emerging light beam, then \\
\[
\begin{bmatrix}
    S_0'      \\[6pt]
    S_1'      \\[6pt]   
    S_2'      \\[6pt]
    S_3'            
\end{bmatrix}
= 
\textbf{M}\begin{bmatrix}
    S_0      \\[6pt]
    S_1      \\[6pt]   
    S_2      \\[6pt]
    S_3           
\end{bmatrix} 
\]
Here, \textbf{M} is the 4x4 matrix known as the Mueller Matrix.

\subsection{Interpretation of the Stokes parameters}
Suppose we have 4 detectors, 3 with polarizers in front of them. Then, 
\begin{itemize}
    \itemsep-0.2em 
    \item \#0 detects total irradiance ($I_0$)
    \item \#1 detects horizontally polarized irradiance ($I_1$)
    \item \#2 detects +45° polarized irradiance ($I_2$)
    \item \#3 detects right circularly polarized irradiance ($I_3$)
\end{itemize}

\newpage
\noindent The stokes parameters are defined in terms of $I_is$ as follows - \\
    $S_0 = I_0$ \\[3pt]
    $S_1 = 2I_1 - I_0$ \\[3pt]
    $S_2 = 2I_2 - I_0$ \\[3pt]
    $S_3 = 2I_3 - I_0$ \\

\noindent $S_0$ denotes the total irradiance, $S_1$ denotes the excess in intensity of light transmitted by a horizontal polarizer over light transmitted by a vertical polarizer, $S_2$ denotes the excess in intensity of light transmitted by a 45° polarizer over light transmitted by a 135° polarizer and $S_3$ denotes the excess in intensity of light transmitted by a RCP filter over light transmitted by a LCP filter. If the light is completely unpolarized, all these excess quantities will be zero.
\bigskip

\noindent We define the degree of polarization in terms of the stokes parameters as follows -
\vspace{-3mm}
\begin{center}
    Degree of Polarization = $\frac{(S_1^2 + S_2^2 + S_3^2)^{1/2}}{S_0}$
\end{center}

\subsection{Polarizing elements}
When a light beam interacts with matter, its polarization state can change in four ways:
\begin{itemize}
    \itemsep-0.2em 
    \item The amplitude of its orthogonal components. This is done with the help of Polarizers/diattenuators.
    \item The relative phase between the orthogonal components. This is done by Phase Shifters (also called waveplates)
    \item The vibrational directions of its orthogonal components. This is achieved by rotators
    \item Energy transfer from polarized states to unpolarized states. Depolarizers are employed to achieve this task,
\end{itemize}



\subsection{Difference between the Stokes vectors and the Jones vectors}
While both Stokes vectors and Jones vectors are used to describe the polarization of light, there are some major differences between them. Light which is unpolarized or partially polarized must be treated with Mueller calculus, whereas fully polarized light can be treated either with Mueller calculus or the Jones calculus. \\
In other words, Stokes vectors and Mueller matrices operate on intensities and their differences, while the Jones matrices operate on the amplitudes of the electric field. 

\section{Conventional Mueller Matrix Polarimetry}
The Mueller polarimeter is a polarimeter used in measuring polarization properties. In a light polarization model, a sample (transmitting/scattering) can be represented by a 4x4 Mueller matrix. In a typical setup, the sample is sequentially illuminated with different input Stokes vectors, which are generated by passing the light from a source through a polarization state generator (PSG), comprised of a polarizer, and quarter wave plate. A complete polarization state generator (PSG) transforms the incoming light, which is unpolarized or has a constant polarization, into a beam of known polarization state. When polarized light travels in the inverse direction, the polarization generator extinguishes a beam with the orthogonal polarization. In this case the PSG works as a polarization state analyzer (PSA). The scattered light after interaction with the sample is passed through the polarization state analyzer (PSA) and finally registered by a detector (e.g., CCD camera, photodiode, etc).

\bigskip
\noindent To calculate the tissue Mueller matrix M, different combinations of input and out polarization states have been utilized. The frequently used input Stokes vectors are: linear horizontal, vertical, +45°, and right circular, while each input state is analyzed with four (six) different output Stokes vectors, including linear horizontal, vertical, +45°, and right circular (linear -45°, left circular). These 16 (24/36) combinations of Stokes vectors are used to calculate the Mueller matrix M of the sample. 

\begin{figure}[h!]
\centering
\includegraphics[scale=0.9]{mmpsetup.png}
\caption{Schematic diagram of a conventional Mueller matrix polarimetry setup}
\label{fig:universe}
\end{figure}

In the above figure, \textbf{1} is the light source, \textbf{2} is a monochromator, \textbf{3} and \textbf{9} are polarizers, \textbf{4} and \textbf{8} are quarter wave plates, \textbf{5}, \textbf{7} and \textbf{10} are lenses, \textbf{6} is the sample and \textbf{11} is the detector. Elements \textbf{2} to \textbf{5} comprise the PSG, while elements \textbf{7} to \textbf{10} comprise the PSA.

\section{Single DoF polarimetry}
\subsection{Formalism}
This is nothing but the Mueller matrix polarimetry discussed above. In this case, the detectors are not sensitive to the spatial degrees of freedom (DoF) and hence, only resolve the polarization DoFs. \\
We represent the standard basis for the polarization qubits and the spatial qubits as follows - 
\vspace{-3mm}
\begin{center}
    $\ket{0}_{pol} = \textbf{e}_x$ and $\ket{1}_{pol} = \textbf{e}_y$ \\[3mm]
    $\ket{0}_{spa} = \psi_{10}(\textbf{r})$ and $\ket{1}_{spa} = \psi_{01}(\textbf{r})$
\end{center}

\noindent We can then write the complete orthogonal basis of the Hilbert space $\mathcal{H}_2$ using the tensor product $\ket{i, j} = \ket{i}_{pol} \otimes \ket{j}_{spa}$, i.e. 
\begin{center}
    $\ket{0, 0} = \textbf{e}_x\psi_{10}(\textbf{r})$ \\[3mm]
    $\ket{0, 1} = \textbf{e}_x\psi_{01}(\textbf{r})$ \\[3mm]
    $\ket{1, 0} = \textbf{e}_y\psi_{10}(\textbf{r})$ \\[3mm]
    $\ket{1, 1} = \textbf{e}_y\psi_{01}(\textbf{r})$
\end{center}

\noindent The electric field can be represented in $\mathcal{H}_2$ as
\begin{center}
     $\ket{E} = A_{00}\ket{0, 0} + A_{01}\ket{0, 1} + A_{10}\ket{1, 0} + A_{11}\ket{1, 1}$
\end{center}
\noindent If the beam is radially polarized, 
\begin{center}
     $\ket{E} = \dfrac{1}{\sqrt{2}}(\ket{0, 0} + \ket{1, 1})$
\end{center}
We now define the \textbf{coherency matrix} $\rho$ as follows - 
\begin{center}
     $\ket{E}\bra{E} = \rho = 
     \begin{bmatrix}
    A_{00}A_{00}^* && A_{00}A_{01}^* && A_{00}A_{10}^* && A_{00}A_{11}^*  \\[6pt]
    A_{01}A_{00}^* && A_{01}A_{01}^* && A_{01}A_{10}^* && A_{01}A_{11}^*  \\[6pt] 
    A_{10}A_{00}^* && A_{10}A_{01}^* && A_{10}A_{10}^* && A_{10}A_{11}^*  \\[6pt]
    A_{11}A_{00}^* && A_{11}A_{01}^* && A_{11}A_{10}^* && A_{11}A_{11}^*  \\[6pt] \end{bmatrix}$

\newpage
For a radially polarized beam,
$\rho = \dfrac{1}{2}\begin{bmatrix}
    1 && 0 && 0 && 1  \\[6pt]
    0 && 0 && 0 && 0  \\[6pt]
    0 && 0 && 0 && 0  \\[6pt]
    1 && 0 && 0 && 1
    \end{bmatrix}$
     
\end{center}

\noindent As mentioned above, the detection scheme in this type of polarimetry cannot resolve the spatial DoFs. In such a case, a proper representation of the beam can be obtained from the coherency matrix by tracing out the unobservable spatial DoFs. We then obtain the 2x2 polarization coherency matrix that encodes all the information about the light beam, regardless of the spatial DoFs.
\begin{align*} 
\rho_{pol} & = tr_{spa}(\rho) \\
 & = AA^{\dagger}
\end{align*}

\noindent In a similar way, the reduced 2 × 2 spatial coherency matrix that encodes all the available information about the spatial modes of the light beam, irrespective of the polarization.
\vspace{-5mm}
\begin{align*} 
\rho_{spa} & = tr_{pol}(\rho) \\
 & = (A^{\dagger}A)^T
\end{align*}
where $A = \begin{bmatrix}
    A_{00} && A_{01} \\
      A_{10} && A_{11}
\end{bmatrix}$
\bigskip
\\ Since $\rho_{pol}$ is hermitian and positive-definite, it admits a Liouville representation of the form 
\begin{center}
$\rho_{pol} = \dfrac{1}{2}\sum_{\mu=0}^{3} S_{\mu}\sigma_{\mu}$
\end{center}
where $S_{\mu}$ are the stokes parameters and $\sigma_{\mu}$ are the Pauli matrices. Since, $\sigma_{\mu}\sigma_{\nu} = 2\delta_{\mu\nu}$, we can write $S_{\mu}$ = tr($\rho_{pol} \sigma_{\mu})$

\subsection{Calculating the Mueller matrix}
Let us consider local linear transformations of the form $T_{pol} \otimes T_{spa}$, i.e. transformations that act on each DoF separately, where $T_i$ is the Jones matrix. Since we are concerned only with the polarization DoF, one can take $T_{spa} = \mathcal{I}_2$, where $\mathcal{I}_2$ is the 2x2 identity matrix. $T_{pol}$ can then be denoted as $T$ itself for convenience sake.
\newpage
\noindent Under the action of T, $E$ transforms to $E'$ as
\begin{center}
    $\ket{E'} = (T \otimes \mathcal{I}_2)\ket{E}$ \\[3mm]
    $\Longrightarrow \ket{E'} = \sum_{i,j=0}^{1} A'_{ij}\ket{i, j}$
\end{center}
Using A' = TA and $\rho_{pol} = A'(A')^{\dagger}$, we get
\begin{center}
    $\rho_{pol}' = TAA^{\dagger}T^{\dagger}$
\end{center}
If the input beam is prepared in 4 different polarization states, each labelled by the index $\alpha \in \{0,1,2,3\}$ and $S_{\nu}(\alpha)$ \& $S'_{\mu}(\alpha)$ are the stokes parameters of the input and output beams respectively , then
\begin{center}
    $S'_{\mu}(\alpha) = \sum_{\nu = 0}^{3} M_{\mu\nu}S_{\nu}(\alpha)$
\end{center}
We then obtain 
\begin{center}
    $M_{\mu\nu} = \dfrac{1}{2}$tr($\sigma_{\mu}T\sigma_{\nu}T^{\dagger}$)
\end{center}
This gives us 16 equations. Noting that $M_{\mu\nu}$ are the elements of the Mueller matrix \textbf{M}, \textbf{M} is finally calculated.

\section{Two-DoF Polarimetry}
In the two-DoF polarimetry, the detector can resolve the polarization DoFs as well as the spatial DoFs. The coherency matrix in this case can be written as
\begin{center}
    $\rho = \dfrac{1}{4}\sum_{\mu,\nu = 0}^{3} S_{\mu\nu} (\sigma_{\mu} \otimes \sigma_{\nu})$
\end{center}
which implies that $S_{\mu\nu} =$ tr[$\rho(\sigma_{\mu} \otimes \sigma_{\nu})$]
The spatial and polarization DoFs are encoded in the same radially polarized beam of light. For such a beam, the 2 DoFs stokes parameters can be written as
\begin{center}
    $S_{\mu\nu} = \lambda_{\mu}\delta_{\mu\nu}$, where $\lambda_{\mu} = \{1,1,-1,1\}$
\end{center}
Under the action of $\mathcal{T}$ similar to the single DoF case (we assume that $\mathcal{T}$ does not vary over the cross section of the input beam), $\rho$ transforms as 
\begin{center}
    $\rho' = (\mathcal{T} \otimes \mathcal{I}_2)\rho(\mathcal{T}^{\dagger} \otimes \mathcal{I}_2)$
\end{center}
\newpage
Combining this relation and $S'_{\mu\nu} =$ tr[$\rho'(\sigma_{\mu} \otimes \sigma_{\nu})$], we get
\begin{center}
    $S'_{\mu\nu} = \dfrac{1}{4}\sum_{\alpha = 0}^{3}\lambda_{\alpha}tr(\sigma_{\mu}\mathcal{T}\sigma_{\alpha}\mathcal{T}^{\dagger})tr(\sigma_{\alpha}\sigma_{\nu})$\\[3mm]
    i.e. $S'_{\mu\nu} = M_{\mu\nu}\lambda_{\mu}$
\end{center}
Since $\lambda_{\nu}$ = 1 for $\nu$ = 0,1,3 and $\lambda_{\nu}$ = -1 for $\nu$ = 2, we directly obtain
\vspace{-10mm}
\begin{center}
    \[
    M_{\mu\nu}= 
\begin{cases}
     S'_{\mu\nu},& \text{if } \nu = 2\\
     -S'_{\mu\nu},& \text{if } \nu = 0,1,3\\
\end{cases}
\]
\end{center}

An important observation here is that in case of the single DoF polarimetry, the polarization state of the light beam was pre-selected before the interaction with the sample, due to which the sample could be probed only by a single polarization state at a time. However in case of the two DoF polarimetry, the polarization state was post-selected after the interaction with the sample, thereby probing the sample by all the possible polarization states at once. 
\end{document}
